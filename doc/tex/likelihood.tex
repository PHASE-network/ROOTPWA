\section{Likelihood Function}
\label{sec:likelihood}

% Fabian:
% rewrite - ln L as (complex vector) x (hermitian matrix) x (complex vector)^*
% instead of complicated construct with 3 sums
% the summation schemes (coherent/incoherent) can be defined via the matrix

\ROOTPWA minimizes the following negative log-likelihood function:
\begin{multline}
  \label{eq:likelihood_function}
  -\ln \mathcal{L}
  = -\sum_{i = 1}^{N} \ln \sBrk[4]{\sum_r^{N_r} \sum_{\refl = \pm 1}
    \Abs[3]{\sum_a^{N^\refl_\text{waves}} \mathcal{T}_a^{r \refl}\, \Psi_a^\refl(\tau_i)}^2 + \mathcal{T}_\text{flat}^2} \\
  + \sBrk[4]{\sum_r^{N_r} \sum_{\refl = \pm 1}
    \sum_{a, b}^{N^\refl_\text{waves}} \mathcal{T}_a^{r \refl}\, \mathcal{T}_b^{r \refl *}\,
    \underbrace{\int\!\! \dif{\phi_3(\tau)}\, \eta(\tau)\,
      \Psi_a^\refl(\tau)\, \Psi_b^{\refl *}(\tau)}_{\displaystyle \equiv I^\refl_{a b}} + \mathcal{T}_\text{flat}^2\, \eta_\text{tot}}
\end{multline}
Here $i$ is the event index, $r$ the rank index, $N_r$ the rank of the
spin-density matrix, and \refl the reflectivity.  The indices
$\cbrk{a}_\refl$ and $\cbrk{b}_\refl$ enumerate the
$N^\refl_\text{waves}$ waves with reflectivity \refl. They contain the
\PX quantum numbers \IGJPCM and the decay specification.  The
$\Psi_a^\refl(\tau; \mThreePi)$ are the phase-space normalized decay
amplitudes
\begin{equation}
  \label{eq:decay_amplitude_norm}
  \Psi_a^\refl(\tau; \mThreePi) \equiv \frac{\widebar{\Psi}\vphantom{\Psi}_a^\refl(\tau; \mThreePi)}
  {\sqrt{\int\! \dif{\phi_3(\tau')} \Abs[1]{\widebar{\Psi}\vphantom{\Psi}_a^\refl(\tau'; \mThreePi)}^2}},
\end{equation}
that depend on the phase space variables $\tau$ and \mThreePi.  In
\cref{eq:decay_amplitude_norm} $\dif{\phi_3(\tau)}$ is the
differential three-body phase-space element and
$\widebar{\Psi}\vphantom{\Psi}_a^\refl(\tau; \mThreePi)$ are the
amplitudes as calculated by the ROOTPWA code.  The normalization of
the decay amplitudes also sets the scale for the transition amplitudes
$\mathcal{T}_a^{r \refl}(\mThreePi, \tpr)$, which are the free
parameters \wrt which \cref{eq:likelihood_function} is minimized.  The
transition amplitude $\mathcal{T}_\text{flat}$ of the isotropic
\enquote{flat} wave is purely real and the corresponding decay
amplitude is unity.  The complex-valued integral matrix
$I^\refl_{a b}$, which is independent of the transition amplitudes,
includes the acceptance $\eta(\tau)$. It is calculated using the Monte
Carlo method:
\begin{equation}
  \label{eq:acc_integral}
  I^\refl_{a b} = \sum_{i = 1}^{N_\text{MC}^\text{acc}} \widebar{\Psi}_a^\refl(\tau_i)\, \widebar{\Psi}_b^{\refl *}(\tau_i)
\end{equation}
where the sum is only over the accepted Monte Carlo events.  The total
phase-space acceptance $\eta_\text{tot}$ is given by the ratio of
accepted and generated MC events
\begin{equation}
  \eta_\text{tot} = \frac{N_\text{MC}^\text{acc}}{N_\text{MC}^\text{gen}}
\end{equation}


\section{Gradient of the Likelihood Function}
\label{sec:likelihood_gradient}

In order to calculate the first partial derivatives of the negative
log-likelihood function \cref{eq:likelihood_function} \wrt its
arguments $\mathcal{T}_a^{r \refl}$ it is advantageous to expand the
absolute squares terms:
\begin{multline}
  \label{eq:likelihood_function_expanded}
  -\ln \mathcal{L}
  = -\sum_{i = 1}^{N} \ln \sBrk[4]{\sum_r^{N_r} \sum_{\refl = \pm 1}
    \sum_{a, b}^{N^\refl_\text{waves}} \mathcal{T}_a^{r \refl}\, \mathcal{T}_b^{r \refl *}\,
    \underbrace{\Psi_a^\refl(\tau_i)\, \Psi_b^{\refl *}(\tau_i)}_{\equiv D^\refl_{a b, i}} + \mathcal{T}_\text{flat}^2} \\
  + \sBrk[4]{\sum_r^{N_r} \sum_{\refl = \pm 1}
    \sum_{a, b}^{N^\refl_\text{waves}} \mathcal{T}_a^{r \refl}\, \mathcal{T}_b^{r \refl *}\, I^\refl_{a b} + \mathcal{T}_\text{flat}^2\, \eta_\text{tot}}
\end{multline}
where the $D^\refl_{a b, i}$ stand just for some complex numbers
calculated from the decay amplitudes for every event.

In \cref{eq:likelihood_function_expanded} we have two terms of the
form
\begin{equation}
  \label{eq:likelihood_function_sum_term}
  A_z \equiv
  \sum_r^{N_r} \sum_{\refl = \pm 1} \sum_{a, b}^{N^\refl_\text{waves}} \mathcal{T}_a^{r \refl}\, \mathcal{T}_b^{r \refl *}\, z^\refl_{a b}
  + \mathcal{T}_\text{flat}^2\, z_\text{flat}
\end{equation}

The gradient of the likelihood function contains two kinds of
elements: partial derivatives \wrt the real and the imaginary parts of
the $\cBrk{\mathcal{T}_a^{r \refl}}$. Defining
\begin{equation}
  \mathcal{T}_a^{r \refl} \equiv x_a^{r \refl} + \i y_a^{r \refl}
  \qquad\text{and}\qquad
  z^\refl_{a b} \equiv u^\refl_{a b} + \i v^\refl_{a b}
\end{equation}
with all $x$, $y$, $u$, and $v$ being real numbers, we can calculate
the derivative of $A_z$ \wrt the real part of
$\mathcal{T}_{\tilde{a}}^{\tilde{r} \tilde{\refl}}$ is
\begin{equation}
  \label{eq:likelihood_function_sum_term_re_deriv_start}
  \begin{split}
    \dpd{A_z}{{x_{\tilde{a}}^{\tilde{r} \tilde{\refl}}}}
    = \dpd{}{{x_{\tilde{a}}^{\tilde{r} \tilde{\refl}}}}
    \Bigg( \sum_r^{N_r} \sum_{\refl = \pm 1} \sum_{a, b}^{N^\refl_\text{waves}}
      &\sBrk[2]{(x_a^{r \refl}\, x_b^{r \refl} + y_a^{r \refl}\, y_b^{r \refl})\, u^\refl_{a b}
        - (y_a^{r \refl}\, x_b^{r \refl} - x_a^{r \refl}\, y_b^{r \refl})\, v^\refl_{a b}} \\
      + \i &\underbrace{\sBrk[2]{(y_a^{r \refl}\, x_b^{r \refl} - x_a^{r \refl}\, y_b^{r \refl})\, u^\refl_{a b}
        + (x_a^{r \refl}\, x_b^{r \refl} + y_a^{r \refl}\, y_b^{r \refl})\, v^\refl_{a b}}}_{\equiv B^{r \refl}_{a b}}
      + \mathcal{T}_\text{flat}^2 \Bigg)
  \end{split}
\end{equation}
One can show that $\sum_{a, b}^{N^\refl_\text{waves}} B^{r \refl}_{a b} = 0$~\footnote{%
%
  We use the fact that $z^\refl_{a b}$ is a hermitian matrix with real
  diagonal elements so that
  \begin{equation}
    \label{eq:z_hermitian}
    u^{\refl}_{a b} = u^{\refl}_{b a},
    \qquad
    v^{\refl}_{a b} = - v^{\refl}_{b a}
    \qquad\text{and}\qquad
    v^{\refl}_{a a} = 0
  \end{equation}
  Therefore
  \begin{align*}
    \sum_{a, b}^{N^\refl_\text{waves}} B^{r \refl}_{a b}
    &= \sum_{a, b}^{N^\refl_\text{waves}}
    \sBrk[2]{(y_a^{r \refl}\, x_b^{r \refl} - x_a^{r \refl}\, y_b^{r \refl})\, u^\refl_{a b}
      + (x_a^{r \refl}\, x_b^{r \refl} + y_a^{r \refl}\, y_b^{r \refl})\, v^\refl_{a b}} \\
    &= \underbrace{\sum_{a, b}^{N^\refl_\text{waves}} y_a^{r \refl}\, x_b^{r \refl}\, u^\refl_{a b}
                 - \sum_{b, a}^{N^\refl_\text{waves}} y_b^{r \refl}\, x_a^{r \refl}\, u^\refl_{b a}}_{= 0} \\
    &\qquad + \sum_{\substack{a, b \\ b > a}}^{N^\refl_\text{waves}} (x_a^{r \refl}\, x_b^{r \refl} + y_a^{r \refl}\, y_b^{r \refl})\, v^\refl_{a b}
            + \sum_{\substack{a, b \\ b < a}}^{N^\refl_\text{waves}} (x_a^{r \refl}\, x_b^{r \refl} + y_a^{r \refl}\, y_b^{r \refl})\, v^\refl_{a b} \\
    &=   \sum_{\substack{a, b \\ b > a}}^{N^\refl_\text{waves}} (x_a^{r \refl}\, x_b^{r \refl} + y_a^{r \refl}\, y_b^{r \refl})\, v^\refl_{a b}
       - \sum_{\substack{b, a \\ a > b}}^{N^\refl_\text{waves}} (x_b^{r \refl}\, x_a^{r \refl} + y_b^{r \refl}\, y_a^{r \refl})\, v^\refl_{b a} \\
    &= 0
  \end{align*}
%
} so that
\begin{equation}
  \dpd{A_z}{{x_{\tilde{a}}^{\tilde{r} \tilde{\refl}}}}
  = \sum_r^{N_r} \sum_{\refl = \pm 1} \sum_{a, b}^{N^\refl_\text{waves}}
  \sBrk[3]{\dpd{x_a^{r \refl}}{{x_{\tilde{a}}^{\tilde{r} \tilde{\refl}}}}\, x_b^{r \refl}
    + x_a^{r \refl}\, \dpd{x_b^{r \refl}}{{x_{\tilde{a}}^{\tilde{r} \tilde{\refl}}}}}\, u^\refl_{a b}
  - \sBrk[3]{y_a^{r \refl}\, \dpd{x_b^{r \refl}}{{x_{\tilde{a}}^{\tilde{r} \tilde{\refl}}}}
    - \dpd{x_a^{r \refl}}{{x_{\tilde{a}}^{\tilde{r} \tilde{\refl}}}}\, y_b^{r \refl}}\, v^\refl_{a b}
\end{equation}
Since
\begin{equation}
  \label{eq:likelihood_function_par_deriv}
  \dpd{x_a^{r \refl}}{{x_{\tilde{a}}^{\tilde{r} \tilde{\refl}}}}
  = \delta_{r \tilde{r}}\, \delta_{\refl \tilde{\refl}}\, \delta_{a \tilde{a}}
\end{equation}
the sums over $r$ and \refl collapse as do some of the sums over the waves
\begin{align}
  \dpd{A_z}{{x_{\tilde{a}}^{\tilde{r} \tilde{\refl}}}}
  &= \sum_{a, b}^{N^{\tilde{\refl}}_\text{waves}}
  \sBrk[3]{\delta_{a \tilde{a}}\, x_b^{\tilde{r} \tilde{\refl}}
    + x_a^{\tilde{r} \tilde{\refl}}\, \delta_{b \tilde{a}}}\, u^{\tilde{\refl}}_{a b}
  - \sBrk[3]{y_a^{\tilde{r} \tilde{\refl}}\, \delta_{b \tilde{a}}
    - \delta_{a \tilde{a}}\, y_b^{\tilde{r} \tilde{\refl}}}\, v^{\tilde{\refl}}_{a b} \\
  &= \sum_{b}^{N^{\tilde{\refl}}_\text{waves}} x_b^{\tilde{r} \tilde{\refl}}\, u^{\tilde{\refl}}_{\tilde{a} b}
    + \sum_{a}^{N^{\tilde{\refl}}_\text{waves}} x_a^{\tilde{r} \tilde{\refl}}\, u^{\tilde{\refl}}_{a \tilde{a}}
  - \sum_{a}^{N^{\tilde{\refl}}_\text{waves}} y_a^{\tilde{r} \tilde{\refl}}\, v^{\tilde{\refl}}_{a \tilde{a}}
    + \sum_{b}^{N^{\tilde{\refl}}_\text{waves}} y_b^{\tilde{r} \tilde{\refl}}\, v^{\tilde{\refl}}_{\tilde{a} b} \\
\end{align}
Using the fact that the matrix $z^\refl_{a b}$ is hermitian
[cf. \cref{eq:z_hermitian}], we can combine the sums and finally arrive
at
\begin{align}
  \dpd{A_z}{{x_{\tilde{a}}^{\tilde{r} \tilde{\refl}}}}
  &= \sum_{b}^{N^{\tilde{\refl}}_\text{waves}} x_b^{\tilde{r} \tilde{\refl}}\, u^{\tilde{\refl}}_{b \tilde{a}}
    + \sum_{a}^{N^{\tilde{\refl}}_\text{waves}} x_a^{\tilde{r} \tilde{\refl}}\, u^{\tilde{\refl}}_{a \tilde{a}}
  - \sum_{a}^{N^{\tilde{\refl}}_\text{waves}} y_a^{\tilde{r} \tilde{\refl}}\, v^{\tilde{\refl}}_{a \tilde{a}}
    - \sum_{b}^{N^{\tilde{\refl}}_\text{waves}} y_b^{\tilde{r} \tilde{\refl}}\, v^{\tilde{\refl}}_{b \tilde{a}} \\
  \label{eq:likelihood_function_sum_term_re_deriv}
  &= 2 \sum_b^{N^{\tilde{\refl}}_\text{waves}} \sBrk{x_b^{\tilde{r} \tilde{\refl}}\, u^{\tilde{\refl}}_{b \tilde{a}}
    - y_b^{\tilde{r} \tilde{\refl}}\, v^{\tilde{\refl}}_{b \tilde{a}}}
  = 2 \sum_b^{N^{\tilde{\refl}}_\text{waves}} \Re\sBrk{\mathcal{T}_b^{\tilde{r} \tilde{\refl}}\, z^{\tilde{\refl}}_{b \tilde{a}}}
\end{align}
Note that the sum runs only over the waves with the same reflectivity.

The corresponding derivative \wrt the imaginary part of the transition
amplitude can be derived in an analogous way.  However, from
\cref{eq:likelihood_function_sum_term_re_deriv_start} one sees that
one just needs to exchange the roles of the $x$ and $y$:
\begin{equation}
  \label{eq:likelihood_function_sum_term_im_deriv}
  \pd{A_z}{{y_{\tilde{a}}^{\tilde{r} \tilde{\refl}}}}
  = 2 \sum_b^{N^{\tilde{\refl}}_\text{waves}} \sBrk{y_b^{\tilde{r} \tilde{\refl}}\, u^{\tilde{\refl}}_{b \tilde{a}}
    + x_b^{\tilde{r} \tilde{\refl}}\, v^{\tilde{\refl}}_{b \tilde{a}}}
  = 2 \sum_b^{N^{\tilde{\refl}}_\text{waves}} \Im\sBrk{\mathcal{T}_b^{\tilde{r} \tilde{\refl}}\, z^{\tilde{\refl}}_{b \tilde{a}}}
\end{equation}

Using
\cref{eq:likelihood_function_sum_term_re_deriv,eq:likelihood_function_sum_term_im_deriv}
we can calculate the derivatives of $-\ln \mathcal{L}$ \wrt the real
and imaginary parts of
$\mathcal{T}_{\tilde{a}}^{\tilde{r} \tilde{\refl}}$:
\begin{align}
  -\dpd{\ln \mathcal{L}}{{x_{\tilde{a}}^{\tilde{r} \tilde{\refl}}}}
  &= \dpd{}{{x_{\tilde{a}}^{\tilde{r} \tilde{\refl}}}} \rBrk{-\sum_{i = 1}^{N} \ln \sBrk{A_{D_i}} + A_I} \\
  &= -\sum_{i = 1}^{N} \frac{1}{A_{D_i}}\, \dpd{A_{D_i}}{{x_{\tilde{a}}^{\tilde{r} \tilde{\refl}}}}
  + \dpd{A_I}{{x_{\tilde{a}}^{\tilde{r} \tilde{\refl}}}} \\
  \label{eq:likelihood_function_re_deriv}
  &= -\sum_{i = 1}^{N} \frac{2}{A_{D_i}} \sum_b^{N^{\tilde{\refl}}_\text{waves}} \underbrace{\sBrk{x_b^{\tilde{r} \tilde{\refl}}\, u^{\tilde{\refl}}_{b \tilde{a}, i}
    - y_b^{\tilde{r} \tilde{\refl}}\, v^{\tilde{\refl}}_{b \tilde{a}, i}}}_{= \Re\sBrk{\mathcal{T}_b^{\tilde{r} \tilde{\refl}}\, D^{\tilde{\refl}}_{b \tilde{a}, i}}}
    + 2 \sum_b^{N^{\tilde{\refl}}_\text{waves}} \underbrace{\sBrk{x_b^{\tilde{r} \tilde{\refl}}\, U^{\tilde{\refl}}_{b \tilde{a}}
    - y_b^{\tilde{r} \tilde{\refl}}\, V^{\tilde{\refl}}_{b \tilde{a}}}}_{= \Re\sBrk{\mathcal{T}_b^{\tilde{r} \tilde{\refl}}\, I^{\tilde{\refl}}_{b \tilde{a}}}}
\end{align}
Analogously one gets
\begin{align}
  \label{eq:likelihood_function_im_deriv}
  -\dpd{\ln \mathcal{L}}{{y_{\tilde{a}}^{\tilde{r} \tilde{\refl}}}}
  &= -\sum_{i = 1}^{N} \frac{2}{A_{D_i}} \sum_b^{N^{\tilde{\refl}}_\text{waves}} \underbrace{\sBrk{y_b^{\tilde{r} \tilde{\refl}}\, u^{\tilde{\refl}}_{b \tilde{a}, i}
    + x_b^{\tilde{r} \tilde{\refl}}\, v^{\tilde{\refl}}_{b \tilde{a}, i}}}_{= \Im\sBrk{\mathcal{T}_b^{\tilde{r} \tilde{\refl}}\, D^{\tilde{\refl}}_{b \tilde{a}, i}}}
    + 2 \sum_b^{N^{\tilde{\refl}}_\text{waves}} \underbrace{\sBrk{y_b^{\tilde{r} \tilde{\refl}}\, U^{\tilde{\refl}}_{b \tilde{a}}
    + x_b^{\tilde{r} \tilde{\refl}}\, V^{\tilde{\refl}}_{b \tilde{a}}}}_{= \Im\sBrk{\mathcal{T}_b^{\tilde{r} \tilde{\refl}}\, I^{\tilde{\refl}}_{b \tilde{a}}}}
\end{align}
In both equations we used the definitions
\begin{equation}
  D^\refl_{a b, i} \equiv u^\refl_{a b, i} + \i v^\refl_{a b, i}
  \qquad\text{and}\qquad
  I^\refl_{a b} \equiv U^\refl_{a b} + \i V^\refl_{a b}
\end{equation}


\section{Hessian Matrix of the Likelihood Function}
\label{sec:likelihood_gradient}

In order to calculate the matrix of second partial derivatives of the
likelihood function \wrt the real and imaginary parts of the
transition amplitudes we start from the first partial derivatives as
given in
\cref{eq:likelihood_function_re_deriv,eq:likelihood_function_im_deriv}.

The second partial derivatives \wrt to the real parts of the
transition amplitudes
$\mathcal{T}_{\tilde{a}}^{\tilde{r} \tilde{\refl}}$ and
$\mathcal{T}_{\doubletilde{a}}^{\doubletilde{r} \doubletilde{\refl}}$
are
\begin{align}
  -\dmd{\ln \mathcal{L}}{2}{{x_{\tilde{a}}^{\tilde{r} \tilde{\refl}}}}{}{{x_{\doubletilde{a}}^{\doubletilde{r} \doubletilde{\refl}}}}{}
  &= \dpd{}{{x_{\doubletilde{a}}^{\doubletilde{r} \doubletilde{\refl}}}}
    \rBrk{-\sum_{i = 1}^{N} \frac{1}{A_{D_i}}\, \dpd{A_{D_i}}{{x_{\tilde{a}}^{\tilde{r} \tilde{\refl}}}} + \dpd{A_I}{{x_{\tilde{a}}^{\tilde{r} \tilde{\refl}}}}} \\
  &= \sum_{i = 1}^{N} \frac{1}{A_{D_i}^2}\, \dpd{A_{D_i}}{{x_{\doubletilde{a}}^{\doubletilde{r} \doubletilde{\refl}}}}\, \dpd{A_{D_i}}{{x_{\tilde{a}}^{\tilde{r} \tilde{\refl}}}}
    - \sum_{i = 1}^{N} \frac{1}{A_{D_i}}\, \dmd{A_{D_i}}{2}{{x_{\tilde{a}}^{\tilde{r} \tilde{\refl}}}}{}{{x_{\doubletilde{a}}^{\doubletilde{r} \doubletilde{\refl}}}}{}
    + \dmd{A_I}{2}{{x_{\tilde{a}}^{\tilde{r} \tilde{\refl}}}}{}{{x_{\doubletilde{a}}^{\doubletilde{r} \doubletilde{\refl}}}}{}
\end{align}
Using \cref{eq:likelihood_function_sum_term_re_deriv} we know the
first-derivative terms in the equation above and can calculate the
second derivatives with \cref{eq:likelihood_function_par_deriv}:
\begin{align}
  \label{eq:likelihood_function_sum_term_re_re_deriv}
  \dmd{A_z}{2}{{x_{\tilde{a}}^{\tilde{r} \tilde{\refl}}}}{}{{x_{\doubletilde{a}}^{\doubletilde{r} \doubletilde{\refl}}}}{}
  &= \dpd{}{{x_{\doubletilde{a}}^{\doubletilde{r} \doubletilde{\refl}}}}
    \rBrk{2 \sum_b^{N^{\tilde{\refl}}_\text{waves}} \sBrk{x_b^{\tilde{r} \tilde{\refl}}\, u^{\tilde{\refl}}_{b \tilde{a}}
    - y_b^{\tilde{r} \tilde{\refl}}\, v^{\tilde{\refl}}_{b \tilde{a}}}} \\
  \label{eq:likelihood_function_sum_term_re_re_deriv}
  &= 2 \delta_{\tilde{r} \doubletilde{r}}\, \delta_{\tilde{\refl} \doubletilde{\refl}}\, u^{\tilde{\refl}}_{\doubletilde{a} \tilde{a}}
\end{align}
Therefore
\begin{align}
  \label{eq:likelihood_function_re_re_deriv}
  -\dmd{\ln \mathcal{L}}{2}{{x_{\tilde{a}}^{\tilde{r} \tilde{\refl}}}}{}{{x_{\doubletilde{a}}^{\doubletilde{r} \doubletilde{\refl}}}}{}
  = &\sum_{i = 1}^{N} \frac{4}{A_{D_i}^2}
  \rBrk[4]{\sum_b^{N^{\doubletilde{\refl}}_\text{waves}}
  \underbrace{\sBrk{x_b^{\doubletilde{r} \doubletilde{\refl}}\, u^{\doubletilde{\refl}}_{b \doubletilde{a}, i}
    - y_b^{\doubletilde{r} \doubletilde{\refl}}\, v^{\doubletilde{\refl}}_{b \doubletilde{a}, i}}}_{%
     = \Re\sBrk{\mathcal{T}_b^{\doubletilde{r} \doubletilde{\refl}}\, D^{\doubletilde{\refl}}_{b \doubletilde{a}, i}}}}
  \rBrk[4]{\sum_b^{N^{\tilde{\refl}}_\text{waves}}
  \underbrace{\sBrk{x_b^{\tilde{r} \tilde{\refl}}\, u^{\tilde{\refl}}_{b \tilde{a}, i}
    - y_b^{\tilde{r} \tilde{\refl}}\, v^{\tilde{\refl}}_{b \tilde{a}, i}}}_{%
    = \Re\sBrk{\mathcal{T}_b^{\tilde{r} \tilde{\refl}}\, D^{\tilde{\refl}}_{b \tilde{a}, i}}}} \\
  &+ 2 \delta_{\tilde{r} \doubletilde{r}}\, \delta_{\tilde{\refl} \doubletilde{\refl}}\,
  \rBrk[4]{-\sum_{i = 1}^{N} \frac{u^{\tilde{\refl}}_{\doubletilde{a} \tilde{a}, i}}{A_{D_i}} + U^{\tilde{\refl}}_{\doubletilde{a} \tilde{a}}}
\end{align}
